% Options for packages loaded elsewhere
\PassOptionsToPackage{unicode}{hyperref}
\PassOptionsToPackage{hyphens}{url}
\PassOptionsToPackage{dvipsnames,svgnames,x11names}{xcolor}
%
\documentclass[
]{article}

\usepackage{amsmath,amssymb}
\usepackage{iftex}
\ifPDFTeX
  \usepackage[T1]{fontenc}
  \usepackage[utf8]{inputenc}
  \usepackage{textcomp} % provide euro and other symbols
\else % if luatex or xetex
  \usepackage{unicode-math}
  \defaultfontfeatures{Scale=MatchLowercase}
  \defaultfontfeatures[\rmfamily]{Ligatures=TeX,Scale=1}
\fi
\usepackage{lmodern}
\ifPDFTeX\else  
    % xetex/luatex font selection
\fi
% Use upquote if available, for straight quotes in verbatim environments
\IfFileExists{upquote.sty}{\usepackage{upquote}}{}
\IfFileExists{microtype.sty}{% use microtype if available
  \usepackage[]{microtype}
  \UseMicrotypeSet[protrusion]{basicmath} % disable protrusion for tt fonts
}{}
\makeatletter
\@ifundefined{KOMAClassName}{% if non-KOMA class
  \IfFileExists{parskip.sty}{%
    \usepackage{parskip}
  }{% else
    \setlength{\parindent}{0pt}
    \setlength{\parskip}{6pt plus 2pt minus 1pt}}
}{% if KOMA class
  \KOMAoptions{parskip=half}}
\makeatother
\usepackage{xcolor}
\setlength{\emergencystretch}{3em} % prevent overfull lines
\setcounter{secnumdepth}{5}
% Make \paragraph and \subparagraph free-standing
\makeatletter
\ifx\paragraph\undefined\else
  \let\oldparagraph\paragraph
  \renewcommand{\paragraph}{
    \@ifstar
      \xxxParagraphStar
      \xxxParagraphNoStar
  }
  \newcommand{\xxxParagraphStar}[1]{\oldparagraph*{#1}\mbox{}}
  \newcommand{\xxxParagraphNoStar}[1]{\oldparagraph{#1}\mbox{}}
\fi
\ifx\subparagraph\undefined\else
  \let\oldsubparagraph\subparagraph
  \renewcommand{\subparagraph}{
    \@ifstar
      \xxxSubParagraphStar
      \xxxSubParagraphNoStar
  }
  \newcommand{\xxxSubParagraphStar}[1]{\oldsubparagraph*{#1}\mbox{}}
  \newcommand{\xxxSubParagraphNoStar}[1]{\oldsubparagraph{#1}\mbox{}}
\fi
\makeatother

\usepackage{color}
\usepackage{fancyvrb}
\newcommand{\VerbBar}{|}
\newcommand{\VERB}{\Verb[commandchars=\\\{\}]}
\DefineVerbatimEnvironment{Highlighting}{Verbatim}{commandchars=\\\{\}}
% Add ',fontsize=\small' for more characters per line
\usepackage{framed}
\definecolor{shadecolor}{RGB}{241,243,245}
\newenvironment{Shaded}{\begin{snugshade}}{\end{snugshade}}
\newcommand{\AlertTok}[1]{\textcolor[rgb]{0.68,0.00,0.00}{#1}}
\newcommand{\AnnotationTok}[1]{\textcolor[rgb]{0.37,0.37,0.37}{#1}}
\newcommand{\AttributeTok}[1]{\textcolor[rgb]{0.40,0.45,0.13}{#1}}
\newcommand{\BaseNTok}[1]{\textcolor[rgb]{0.68,0.00,0.00}{#1}}
\newcommand{\BuiltInTok}[1]{\textcolor[rgb]{0.00,0.23,0.31}{#1}}
\newcommand{\CharTok}[1]{\textcolor[rgb]{0.13,0.47,0.30}{#1}}
\newcommand{\CommentTok}[1]{\textcolor[rgb]{0.37,0.37,0.37}{#1}}
\newcommand{\CommentVarTok}[1]{\textcolor[rgb]{0.37,0.37,0.37}{\textit{#1}}}
\newcommand{\ConstantTok}[1]{\textcolor[rgb]{0.56,0.35,0.01}{#1}}
\newcommand{\ControlFlowTok}[1]{\textcolor[rgb]{0.00,0.23,0.31}{\textbf{#1}}}
\newcommand{\DataTypeTok}[1]{\textcolor[rgb]{0.68,0.00,0.00}{#1}}
\newcommand{\DecValTok}[1]{\textcolor[rgb]{0.68,0.00,0.00}{#1}}
\newcommand{\DocumentationTok}[1]{\textcolor[rgb]{0.37,0.37,0.37}{\textit{#1}}}
\newcommand{\ErrorTok}[1]{\textcolor[rgb]{0.68,0.00,0.00}{#1}}
\newcommand{\ExtensionTok}[1]{\textcolor[rgb]{0.00,0.23,0.31}{#1}}
\newcommand{\FloatTok}[1]{\textcolor[rgb]{0.68,0.00,0.00}{#1}}
\newcommand{\FunctionTok}[1]{\textcolor[rgb]{0.28,0.35,0.67}{#1}}
\newcommand{\ImportTok}[1]{\textcolor[rgb]{0.00,0.46,0.62}{#1}}
\newcommand{\InformationTok}[1]{\textcolor[rgb]{0.37,0.37,0.37}{#1}}
\newcommand{\KeywordTok}[1]{\textcolor[rgb]{0.00,0.23,0.31}{\textbf{#1}}}
\newcommand{\NormalTok}[1]{\textcolor[rgb]{0.00,0.23,0.31}{#1}}
\newcommand{\OperatorTok}[1]{\textcolor[rgb]{0.37,0.37,0.37}{#1}}
\newcommand{\OtherTok}[1]{\textcolor[rgb]{0.00,0.23,0.31}{#1}}
\newcommand{\PreprocessorTok}[1]{\textcolor[rgb]{0.68,0.00,0.00}{#1}}
\newcommand{\RegionMarkerTok}[1]{\textcolor[rgb]{0.00,0.23,0.31}{#1}}
\newcommand{\SpecialCharTok}[1]{\textcolor[rgb]{0.37,0.37,0.37}{#1}}
\newcommand{\SpecialStringTok}[1]{\textcolor[rgb]{0.13,0.47,0.30}{#1}}
\newcommand{\StringTok}[1]{\textcolor[rgb]{0.13,0.47,0.30}{#1}}
\newcommand{\VariableTok}[1]{\textcolor[rgb]{0.07,0.07,0.07}{#1}}
\newcommand{\VerbatimStringTok}[1]{\textcolor[rgb]{0.13,0.47,0.30}{#1}}
\newcommand{\WarningTok}[1]{\textcolor[rgb]{0.37,0.37,0.37}{\textit{#1}}}

\providecommand{\tightlist}{%
  \setlength{\itemsep}{0pt}\setlength{\parskip}{0pt}}\usepackage{longtable,booktabs,array}
\usepackage{calc} % for calculating minipage widths
% Correct order of tables after \paragraph or \subparagraph
\usepackage{etoolbox}
\makeatletter
\patchcmd\longtable{\par}{\if@noskipsec\mbox{}\fi\par}{}{}
\makeatother
% Allow footnotes in longtable head/foot
\IfFileExists{footnotehyper.sty}{\usepackage{footnotehyper}}{\usepackage{footnote}}
\makesavenoteenv{longtable}
\usepackage{graphicx}
\makeatletter
\newsavebox\pandoc@box
\newcommand*\pandocbounded[1]{% scales image to fit in text height/width
  \sbox\pandoc@box{#1}%
  \Gscale@div\@tempa{\textheight}{\dimexpr\ht\pandoc@box+\dp\pandoc@box\relax}%
  \Gscale@div\@tempb{\linewidth}{\wd\pandoc@box}%
  \ifdim\@tempb\p@<\@tempa\p@\let\@tempa\@tempb\fi% select the smaller of both
  \ifdim\@tempa\p@<\p@\scalebox{\@tempa}{\usebox\pandoc@box}%
  \else\usebox{\pandoc@box}%
  \fi%
}
% Set default figure placement to htbp
\def\fps@figure{htbp}
\makeatother

\makeatletter
\@ifpackageloaded{caption}{}{\usepackage{caption}}
\AtBeginDocument{%
\ifdefined\contentsname
  \renewcommand*\contentsname{Table of contents}
\else
  \newcommand\contentsname{Table of contents}
\fi
\ifdefined\listfigurename
  \renewcommand*\listfigurename{List of Figures}
\else
  \newcommand\listfigurename{List of Figures}
\fi
\ifdefined\listtablename
  \renewcommand*\listtablename{List of Tables}
\else
  \newcommand\listtablename{List of Tables}
\fi
\ifdefined\figurename
  \renewcommand*\figurename{Figure}
\else
  \newcommand\figurename{Figure}
\fi
\ifdefined\tablename
  \renewcommand*\tablename{Table}
\else
  \newcommand\tablename{Table}
\fi
}
\@ifpackageloaded{float}{}{\usepackage{float}}
\floatstyle{ruled}
\@ifundefined{c@chapter}{\newfloat{codelisting}{h}{lop}}{\newfloat{codelisting}{h}{lop}[chapter]}
\floatname{codelisting}{Listing}
\newcommand*\listoflistings{\listof{codelisting}{List of Listings}}
\makeatother
\makeatletter
\makeatother
\makeatletter
\@ifpackageloaded{caption}{}{\usepackage{caption}}
\@ifpackageloaded{subcaption}{}{\usepackage{subcaption}}
\makeatother

\usepackage{bookmark}

\IfFileExists{xurl.sty}{\usepackage{xurl}}{} % add URL line breaks if available
\urlstyle{same} % disable monospaced font for URLs
\hypersetup{
  pdftitle={CTMC and Queueing Lab: Vaccine Clinic Operations},
  colorlinks=true,
  linkcolor={blue},
  filecolor={Maroon},
  citecolor={Blue},
  urlcolor={Blue},
  pdfcreator={LaTeX via pandoc}}


\title{CTMC and Queueing Lab: Vaccine Clinic Operations}
\author{}
\date{}

\begin{document}
\maketitle

\renewcommand*\contentsname{Table of contents}
{
\hypersetup{linkcolor=}
\setcounter{tocdepth}{3}
\tableofcontents
}

\section{Introduction}\label{introduction}

In this short lab, we'll explore how Continuous-Time Markov Chains
(CTMCs) and Queueing Theory apply to healthcare settings using realistic
examples. The goal is to build intuition and hands-on familiarity
through code you can run and interpret.

We'll use the example of a vaccine clinic with limited observation
space, followed by a queueing system involving patients waiting for
service.

\begin{center}\rule{0.5\linewidth}{0.5pt}\end{center}

\section{Part I: CTMC -- Vaccine Observation
Room}\label{part-i-ctmc-vaccine-observation-room}

In a rural health facility, adults are being vaccinated against rubella.
After receiving their shots, each patient must be observed for side
effects in a dedicated room. Due to space limits, only two patients can
be monitored at a time. Initially, patients arrive approximately once
every minute. If there is space, they are admitted; otherwise, they are
turned away. When alone in the room, each person is monitored for 10
minutes. If two are observed together, both are monitored for 8 minutes
due to resource constraints.

\subsection{Task 1. Construct the CTMC Generator
Matrix}\label{task-1.-construct-the-ctmc-generator-matrix}

Can you build a model of the possible occupancy changes in this small
clinic? Think about transitions between 0, 1, or 2 patients and how
often they happen.

\begin{Shaded}
\begin{Highlighting}[]
\NormalTok{Q }\OtherTok{\textless{}{-}} \FunctionTok{matrix}\NormalTok{(}\FunctionTok{c}\NormalTok{(}
\NormalTok{  \_\_, \_\_, \_\_,}
\NormalTok{  \_\_, \_\_, \_\_,}
\NormalTok{  \_\_, \_\_, \_\_}
\NormalTok{), }\AttributeTok{nrow =} \DecValTok{3}\NormalTok{, }\AttributeTok{byrow =} \ConstantTok{TRUE}\NormalTok{)}
\NormalTok{Q}
\end{Highlighting}
\end{Shaded}

\subsection{Task 2. How much time does the clinic spend in each
state?}\label{task-2.-how-much-time-does-the-clinic-spend-in-each-state}

Over a long day, how much time does the clinic spend in each state ---
0, 1, or 2 patients in the observation room?

\begin{Shaded}
\begin{Highlighting}[]
\NormalTok{steady\_state }\OtherTok{\textless{}{-}} \ControlFlowTok{function}\NormalTok{(Q) \{}
\NormalTok{  A }\OtherTok{\textless{}{-}} \FunctionTok{t}\NormalTok{(Q) }\CommentTok{\# Transpose Q so rows become equations}
\NormalTok{  A[}\FunctionTok{nrow}\NormalTok{(A), ] }\OtherTok{\textless{}{-}} \FunctionTok{rep}\NormalTok{(}\DecValTok{1}\NormalTok{, }\FunctionTok{ncol}\NormalTok{(Q))  }\CommentTok{\# Replace last row with 1s for normalization}
\NormalTok{  b }\OtherTok{\textless{}{-}} \FunctionTok{c}\NormalTok{(\_\_, \_\_, \_\_) }\CommentTok{\# RHS of the linear system (0s and 1}
  \FunctionTok{solve}\NormalTok{(\_\_, \_\_) }\CommentTok{\#Solve the linear system A * pi = b}
\NormalTok{\}}

\NormalTok{pi }\OtherTok{\textless{}{-}} \FunctionTok{steady\_state}\NormalTok{(Q)}
\FunctionTok{names}\NormalTok{(pi) }\OtherTok{\textless{}{-}} \FunctionTok{paste0}\NormalTok{(}\StringTok{"State "}\NormalTok{, }\DecValTok{0}\SpecialCharTok{:}\DecValTok{2}\NormalTok{)}
\NormalTok{pi}
\end{Highlighting}
\end{Shaded}

\subsection{Task 3: How Does the System Converge to Steady
State?}\label{task-3-how-does-the-system-converge-to-steady-state}

Let's visualize the convergence to the steady-state distribution by
tracking the probability of each state over time.

\subsubsection{Understanding Convergence in a
CTMC}\label{understanding-convergence-in-a-ctmc}

Before we plot how the system converges, let's understand the math
behind it.

In a \textbf{Continuous-Time Markov Chain (CTMC)}, the system moves
between states randomly over continuous time. To know how the system
evolves, we want to compute:

\begin{quote}
\textbf{What is the probability of being in each state after time t?}
\end{quote}

This is given by the formula:

\begin{quote}
\textbf{State distribution at time t = Initial distribution × exp(Q ×
t)}
\end{quote}

\begin{itemize}
\tightlist
\item
  The \textbf{initial distribution} is a row vector (e.g.,
  \texttt{p0\ \textless{}-\ c(1,\ 0,\ 0)} if the system starts in State
  0).
\item
  The \textbf{matrix exponential}, written as \texttt{exp(Q\ ×\ t)} in
  R, gives the \textbf{transition probability matrix} at time
  \texttt{t}.
\item
  Multiplying the two gives us the \textbf{probability distribution
  across states} at that time.
\end{itemize}

We repeat this calculation for many time points (e.g., from 0 to 100
minutes) to see \textbf{how the system changes over time} and eventually
\textbf{settles into a steady state}.

In the next task, you'll:

\begin{itemize}
\tightlist
\item
  Define the time points
\item
  Set the initial condition
\item
  Use a loop to calculate the probability vector at each time
\item
  Plot how the state probabilities evolve
\end{itemize}

Let's implement this now.

\begin{Shaded}
\begin{Highlighting}[]
\CommentTok{\# Define a sequence of time points}
\NormalTok{times }\OtherTok{\textless{}{-}} \FunctionTok{seq}\NormalTok{(\_\_\_, \_\_\_, }\AttributeTok{by =}\NormalTok{ \_\_\_)}

\CommentTok{\# Set the initial state distribution (starts in State 0)}
\NormalTok{p0 }\OtherTok{\textless{}{-}} \FunctionTok{c}\NormalTok{(\_\_\_, \_\_\_, \_\_\_)}

\CommentTok{\# Compute how probabilities evolve over time}
\NormalTok{probs\_over\_time }\OtherTok{\textless{}{-}} \FunctionTok{t}\NormalTok{(}\FunctionTok{sapply}\NormalTok{(times, }\ControlFlowTok{function}\NormalTok{(t) p0 }\SpecialCharTok{\%*\%} \FunctionTok{expm}\NormalTok{(\_\_\_)))}

\CommentTok{\# Plot the result}
\FunctionTok{matplot}\NormalTok{(times, probs\_over\_time, }\AttributeTok{type =} \StringTok{"l"}\NormalTok{, }\AttributeTok{lty =} \DecValTok{1}\NormalTok{, }\AttributeTok{lwd =} \DecValTok{2}\NormalTok{,}
        \AttributeTok{col =} \FunctionTok{c}\NormalTok{(}\StringTok{"blue"}\NormalTok{, }\StringTok{"orange"}\NormalTok{, }\StringTok{"red"}\NormalTok{),}
        \AttributeTok{ylab =} \StringTok{"Probability"}\NormalTok{, }\AttributeTok{xlab =} \StringTok{"Time"}\NormalTok{,}
        \AttributeTok{main =} \StringTok{"Convergence to Steady State"}\NormalTok{)}

\FunctionTok{legend}\NormalTok{(}\StringTok{"right"}\NormalTok{, }\AttributeTok{legend =} \FunctionTok{c}\NormalTok{(}\StringTok{"State 0"}\NormalTok{, }\StringTok{"State 1"}\NormalTok{, }\StringTok{"State 2"}\NormalTok{),}
        \AttributeTok{col =} \FunctionTok{c}\NormalTok{(}\StringTok{"blue"}\NormalTok{, }\StringTok{"orange"}\NormalTok{, }\StringTok{"red"}\NormalTok{), }\AttributeTok{lty =} \DecValTok{1}\NormalTok{, }\AttributeTok{lwd =} \DecValTok{2}\NormalTok{)}
\end{Highlighting}
\end{Shaded}

Explain: How long does it take for the system to reach equilibrium?
Which state dominates during the transient phase?

\emph{\textbf{Answer:}}

\begin{center}\rule{0.5\linewidth}{0.5pt}\end{center}

\begin{center}\rule{0.5\linewidth}{0.5pt}\end{center}

\begin{center}\rule{0.5\linewidth}{0.5pt}\end{center}

\subsection{Task 4. How many people are typically in the
room?}\label{task-4.-how-many-people-are-typically-in-the-room}

On average, how many people are being observed in the room at any given
time? This gives a sense of how often the room is full or idle.

\begin{Shaded}
\begin{Highlighting}[]
\FunctionTok{sum}\NormalTok{(}\DecValTok{0}\SpecialCharTok{:}\DecValTok{2} \SpecialCharTok{*}\NormalTok{ \_\_)}
\end{Highlighting}
\end{Shaded}

\subsection{Task 5. What if more patients start coming
in?}\label{task-5.-what-if-more-patients-start-coming-in}

Due to a new outreach campaign and the arrival of a donor-funded vaccine
shipment, community interest in rubella vaccination has surged. As a
result, patients are now arriving every 30 minutes, doubling the
previous arrival rate. Observation lasts 20 minutes when one person is
present. If two patients are present, they both stay for 17 minutes.

How does this change the expected number of patients inside the room?

\begin{Shaded}
\begin{Highlighting}[]
\NormalTok{Q2 }\OtherTok{\textless{}{-}} \FunctionTok{matrix}\NormalTok{(}\FunctionTok{c}\NormalTok{(}
\NormalTok{  \_\_, \_\_, \_\_,}
\NormalTok{  \_\_, \_\_, \_\_,}
\NormalTok{  \_\_, \_\_, \_\_}
\NormalTok{), }\AttributeTok{nrow =} \DecValTok{3}\NormalTok{, }\AttributeTok{byrow =} \ConstantTok{TRUE}\NormalTok{)}

\NormalTok{pi2 }\OtherTok{\textless{}{-}} \FunctionTok{steady\_state}\NormalTok{(\_\_)}
\FunctionTok{round}\NormalTok{(pi2, }\DecValTok{4}\NormalTok{)}

\FunctionTok{sum}\NormalTok{(}\DecValTok{0}\SpecialCharTok{:}\DecValTok{2} \SpecialCharTok{*}\NormalTok{ \_\_)  }\CommentTok{\# new expected number}
\end{Highlighting}
\end{Shaded}

\begin{center}\rule{0.5\linewidth}{0.5pt}\end{center}

\section{Part II: Queueing -- Patient Check-in
Desk}\label{part-ii-queueing-patient-check-in-desk}

In the same rural vaccine clinic, patients begin their visit at a
\textbf{check-in desk} staffed by a single nurse. Patients arrive
randomly, on average once every two minutes. Each check-in takes about
1.5 minutes and is handled one at a time. If the nurse is already
helping someone, the remaining patients wait their turn. There is no
upper limit on how many patients can wait in line.

\subsection{Task 6. How busy is the nurse, and how long do patients
wait?}\label{task-6.-how-busy-is-the-nurse-and-how-long-do-patients-wait}

The clinic administrator wants to know: How busy is the nurse? How long
does each patient spend waiting and checking in? And how many people are
typically in the system?

\begin{Shaded}
\begin{Highlighting}[]
\NormalTok{lambda }\OtherTok{\textless{}{-}}\NormalTok{ \_\_  }\CommentTok{\# arrivals per minute}
\NormalTok{mu }\OtherTok{\textless{}{-}}\NormalTok{ \_\_      }\CommentTok{\# service rate per minute (1/1.5)}

\NormalTok{rho }\OtherTok{\textless{}{-}}\NormalTok{ \_\_ }\SpecialCharTok{/}\NormalTok{ \_\_}
\NormalTok{L }\OtherTok{\textless{}{-}}\NormalTok{ \_\_ }\SpecialCharTok{/}\NormalTok{ (}\DecValTok{1} \SpecialCharTok{{-}}\NormalTok{ \_\_)}
\NormalTok{Lq }\OtherTok{\textless{}{-}}\NormalTok{ \_\_}\SpecialCharTok{\^{}}\DecValTok{2} \SpecialCharTok{/}\NormalTok{ (}\DecValTok{1} \SpecialCharTok{{-}}\NormalTok{ \_\_)}
\NormalTok{W }\OtherTok{\textless{}{-}} \DecValTok{1} \SpecialCharTok{/}\NormalTok{ (mu }\SpecialCharTok{*}\NormalTok{ (}\DecValTok{1} \SpecialCharTok{{-}}\NormalTok{ \_\_))}
\NormalTok{Wq }\OtherTok{\textless{}{-}}\NormalTok{ \_\_ }\SpecialCharTok{/}\NormalTok{ (mu }\SpecialCharTok{*}\NormalTok{ (}\DecValTok{1} \SpecialCharTok{{-}}\NormalTok{ \_\_))}

\FunctionTok{list}\NormalTok{(}
  \AttributeTok{Utilization =} \FunctionTok{round}\NormalTok{(rho, }\DecValTok{2}\NormalTok{),}
  \StringTok{"Avg in System (L)"} \OtherTok{=} \FunctionTok{round}\NormalTok{(L, }\DecValTok{2}\NormalTok{),}
  \StringTok{"Avg in Queue (Lq)"} \OtherTok{=} \FunctionTok{round}\NormalTok{(Lq, }\DecValTok{2}\NormalTok{),}
  \StringTok{"Time in System (W)"} \OtherTok{=} \FunctionTok{round}\NormalTok{(W, }\DecValTok{2}\NormalTok{),}
  \StringTok{"Time in Queue (Wq)"} \OtherTok{=} \FunctionTok{round}\NormalTok{(Wq, }\DecValTok{2}\NormalTok{)}
\NormalTok{)}
\end{Highlighting}
\end{Shaded}

\subsection{Task 7. What happens when more patients keep
coming?}\label{task-7.-what-happens-when-more-patients-keep-coming}

Imagine the clinic gets busier over time --- more patients show up
without any increase in staff. Let's see how the average number of
patients and the uncertainty around that number change.

\begin{Shaded}
\begin{Highlighting}[]
\CommentTok{\# Define a sequence of traffic intensity (ρ) values from 0.01 to 0.99}
\NormalTok{rho\_vals }\OtherTok{\textless{}{-}} \FunctionTok{seq}\NormalTok{(}\FloatTok{0.01}\NormalTok{, }\FloatTok{0.99}\NormalTok{, }\AttributeTok{by =} \FloatTok{0.01}\NormalTok{)}

\CommentTok{\# Calculate the average number of patients in the system (L)}
\NormalTok{L\_vals }\OtherTok{\textless{}{-}}\NormalTok{ \_\_\_ }\SpecialCharTok{/}\NormalTok{ (}\DecValTok{1} \SpecialCharTok{{-}}\NormalTok{ \_\_\_)   }\CommentTok{\# FILL IN: Use rho\_vals}

\CommentTok{\# Calculate the variance of the number of patients in the system}
\CommentTok{\# We calculate the variance to understand how unpredictable the system is.}
\CommentTok{\# A high variance means it\textquotesingle{}s harder to estimate how many patients are actually there at any time.}
\NormalTok{Var\_vals }\OtherTok{\textless{}{-}}\NormalTok{ \_\_\_ }\SpecialCharTok{*}\NormalTok{ (}\DecValTok{1} \SpecialCharTok{+}\NormalTok{ \_\_\_ }\SpecialCharTok{{-}}\NormalTok{ \_\_\_}\SpecialCharTok{\^{}}\DecValTok{2}\NormalTok{) }\SpecialCharTok{/}\NormalTok{ (}\DecValTok{1} \SpecialCharTok{{-}}\NormalTok{ \_\_\_)}\SpecialCharTok{\^{}}\DecValTok{2}   \CommentTok{\# FILL IN: Use rho\_vals multiple times}

\CommentTok{\# Plot the results}
\FunctionTok{plot}\NormalTok{(rho\_vals, L\_vals, }\AttributeTok{type =} \StringTok{"l"}\NormalTok{, }\AttributeTok{col =} \StringTok{"red"}\NormalTok{, }\AttributeTok{lwd =} \DecValTok{2}\NormalTok{,}
     \AttributeTok{ylim =} \FunctionTok{c}\NormalTok{(}\DecValTok{0}\NormalTok{, }\DecValTok{40}\NormalTok{), }\AttributeTok{ylab =} \StringTok{"Mean \& Variance in System"}\NormalTok{, }\AttributeTok{xlab =} \StringTok{"Traffic Intensity (ρ)"}\NormalTok{)}
\FunctionTok{lines}\NormalTok{(rho\_vals, Var\_vals, }\AttributeTok{col =} \StringTok{"blue"}\NormalTok{, }\AttributeTok{lwd =} \DecValTok{2}\NormalTok{, }\AttributeTok{lty =} \DecValTok{2}\NormalTok{)}
\FunctionTok{legend}\NormalTok{(}\StringTok{"topleft"}\NormalTok{, }\AttributeTok{legend =} \FunctionTok{c}\NormalTok{(}\StringTok{"Mean"}\NormalTok{, }\StringTok{"Variance"}\NormalTok{), }\AttributeTok{col =} \FunctionTok{c}\NormalTok{(}\StringTok{"red"}\NormalTok{, }\StringTok{"blue"}\NormalTok{), }\AttributeTok{lty =} \FunctionTok{c}\NormalTok{(}\DecValTok{1}\NormalTok{, }\DecValTok{2}\NormalTok{), }\AttributeTok{lwd =} \DecValTok{2}\NormalTok{)}
\end{Highlighting}
\end{Shaded}

Explain: What does the graph suggest about the workload and
predictability of the check-in desk as it gets busier?

\textbf{\emph{Answer}}

\begin{center}\rule{0.5\linewidth}{0.5pt}\end{center}

\begin{center}\rule{0.5\linewidth}{0.5pt}\end{center}

\begin{center}\rule{0.5\linewidth}{0.5pt}\end{center}




\end{document}
