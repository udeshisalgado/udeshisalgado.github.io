% Options for packages loaded elsewhere
\PassOptionsToPackage{unicode}{hyperref}
\PassOptionsToPackage{hyphens}{url}
\PassOptionsToPackage{dvipsnames,svgnames,x11names}{xcolor}
%
\documentclass[
]{article}

\usepackage{amsmath,amssymb}
\usepackage{iftex}
\ifPDFTeX
  \usepackage[T1]{fontenc}
  \usepackage[utf8]{inputenc}
  \usepackage{textcomp} % provide euro and other symbols
\else % if luatex or xetex
  \usepackage{unicode-math}
  \defaultfontfeatures{Scale=MatchLowercase}
  \defaultfontfeatures[\rmfamily]{Ligatures=TeX,Scale=1}
\fi
\usepackage{lmodern}
\ifPDFTeX\else  
    % xetex/luatex font selection
\fi
% Use upquote if available, for straight quotes in verbatim environments
\IfFileExists{upquote.sty}{\usepackage{upquote}}{}
\IfFileExists{microtype.sty}{% use microtype if available
  \usepackage[]{microtype}
  \UseMicrotypeSet[protrusion]{basicmath} % disable protrusion for tt fonts
}{}
\makeatletter
\@ifundefined{KOMAClassName}{% if non-KOMA class
  \IfFileExists{parskip.sty}{%
    \usepackage{parskip}
  }{% else
    \setlength{\parindent}{0pt}
    \setlength{\parskip}{6pt plus 2pt minus 1pt}}
}{% if KOMA class
  \KOMAoptions{parskip=half}}
\makeatother
\usepackage{xcolor}
\setlength{\emergencystretch}{3em} % prevent overfull lines
\setcounter{secnumdepth}{-\maxdimen} % remove section numbering
% Make \paragraph and \subparagraph free-standing
\makeatletter
\ifx\paragraph\undefined\else
  \let\oldparagraph\paragraph
  \renewcommand{\paragraph}{
    \@ifstar
      \xxxParagraphStar
      \xxxParagraphNoStar
  }
  \newcommand{\xxxParagraphStar}[1]{\oldparagraph*{#1}\mbox{}}
  \newcommand{\xxxParagraphNoStar}[1]{\oldparagraph{#1}\mbox{}}
\fi
\ifx\subparagraph\undefined\else
  \let\oldsubparagraph\subparagraph
  \renewcommand{\subparagraph}{
    \@ifstar
      \xxxSubParagraphStar
      \xxxSubParagraphNoStar
  }
  \newcommand{\xxxSubParagraphStar}[1]{\oldsubparagraph*{#1}\mbox{}}
  \newcommand{\xxxSubParagraphNoStar}[1]{\oldsubparagraph{#1}\mbox{}}
\fi
\makeatother


\providecommand{\tightlist}{%
  \setlength{\itemsep}{0pt}\setlength{\parskip}{0pt}}\usepackage{longtable,booktabs,array}
\usepackage{calc} % for calculating minipage widths
% Correct order of tables after \paragraph or \subparagraph
\usepackage{etoolbox}
\makeatletter
\patchcmd\longtable{\par}{\if@noskipsec\mbox{}\fi\par}{}{}
\makeatother
% Allow footnotes in longtable head/foot
\IfFileExists{footnotehyper.sty}{\usepackage{footnotehyper}}{\usepackage{footnote}}
\makesavenoteenv{longtable}
\usepackage{graphicx}
\makeatletter
\newsavebox\pandoc@box
\newcommand*\pandocbounded[1]{% scales image to fit in text height/width
  \sbox\pandoc@box{#1}%
  \Gscale@div\@tempa{\textheight}{\dimexpr\ht\pandoc@box+\dp\pandoc@box\relax}%
  \Gscale@div\@tempb{\linewidth}{\wd\pandoc@box}%
  \ifdim\@tempb\p@<\@tempa\p@\let\@tempa\@tempb\fi% select the smaller of both
  \ifdim\@tempa\p@<\p@\scalebox{\@tempa}{\usebox\pandoc@box}%
  \else\usebox{\pandoc@box}%
  \fi%
}
% Set default figure placement to htbp
\def\fps@figure{htbp}
\makeatother

\makeatletter
\@ifpackageloaded{caption}{}{\usepackage{caption}}
\AtBeginDocument{%
\ifdefined\contentsname
  \renewcommand*\contentsname{Table of contents}
\else
  \newcommand\contentsname{Table of contents}
\fi
\ifdefined\listfigurename
  \renewcommand*\listfigurename{List of Figures}
\else
  \newcommand\listfigurename{List of Figures}
\fi
\ifdefined\listtablename
  \renewcommand*\listtablename{List of Tables}
\else
  \newcommand\listtablename{List of Tables}
\fi
\ifdefined\figurename
  \renewcommand*\figurename{Figure}
\else
  \newcommand\figurename{Figure}
\fi
\ifdefined\tablename
  \renewcommand*\tablename{Table}
\else
  \newcommand\tablename{Table}
\fi
}
\@ifpackageloaded{float}{}{\usepackage{float}}
\floatstyle{ruled}
\@ifundefined{c@chapter}{\newfloat{codelisting}{h}{lop}}{\newfloat{codelisting}{h}{lop}[chapter]}
\floatname{codelisting}{Listing}
\newcommand*\listoflistings{\listof{codelisting}{List of Listings}}
\makeatother
\makeatletter
\makeatother
\makeatletter
\@ifpackageloaded{caption}{}{\usepackage{caption}}
\@ifpackageloaded{subcaption}{}{\usepackage{subcaption}}
\makeatother

\usepackage{bookmark}

\IfFileExists{xurl.sty}{\usepackage{xurl}}{} % add URL line breaks if available
\urlstyle{same} % disable monospaced font for URLs
\hypersetup{
  pdftitle={CTMC and Queueing Lab: Vaccine Clinic Operations},
  colorlinks=true,
  linkcolor={blue},
  filecolor={Maroon},
  citecolor={Blue},
  urlcolor={Blue},
  pdfcreator={LaTeX via pandoc}}


\title{CTMC and Queueing Lab: Vaccine Clinic Operations}
\usepackage{etoolbox}
\makeatletter
\providecommand{\subtitle}[1]{% add subtitle to \maketitle
  \apptocmd{\@title}{\par {\large #1 \par}}{}{}
}
\makeatother
\subtitle{Course Desorption}
\author{}
\date{}

\begin{document}
\maketitle


\section{Who is the course for?}\label{who-is-the-course-for}

This workshop is designed for healthcare operations researchers, supply
chain analysts, and graduate students interested in modeling system
dynamics under uncertainty. Participants will learn how Continuous-Time
Markov Chains (CTMCs) and basic queueing models can be applied to
real-world healthcare logistics. The course assumes basic familiarity
with probability and matrix operations in R (or Python).

\section{Learning Objectives}\label{learning-objectives}

By the end of this session, participants will be able to:

\begin{itemize}
\tightlist
\item
  Understand the key components and assumptions of Continuous-Time
  Markov Chains (CTMCs)
\item
  Construct and analyze generator matrices for small-scale CTMC models
\item
  Calculate steady-state probabilities for CTMCs using linear algebra
\item
  Apply queueing theory to model service systems like clinics and
  pharmacies
\item
  Distinguish between M/M/1, M/M/s, and M/M/s/b systems and assess their
  implications
\item
  Simulate and interpret Markovian queueing systems in R
\item
  Connect model insights to operational decisions in healthcare delivery
  (e.g., capacity planning, congestion, patient flow)
\end{itemize}

\section{Prerequisites}\label{prerequisites}

\begin{itemize}
\tightlist
\item
  Familiarity with random variables and basic probability concepts
\item
  Experience with R or Python for numerical computations and plotting
\item
  No prior knowledge of CTMCs or queueing theory is required
\end{itemize}

\section{Course Topics}\label{course-topics}

\subsection{Section 1: Recap}\label{section-1-recap}

\begin{itemize}
\tightlist
\item
  Discrete-time vs.~continuous-time Markov chains
\item
  Markov property and memorylessness
\end{itemize}

\subsection{Section 2: Foundations of Continuous-Time Markov Chains
(CTMCs)}\label{section-2-foundations-of-continuous-time-markov-chains-ctmcs}

\begin{itemize}
\tightlist
\item
  Exponential timing and transition rates
\item
  The generator matrix and its interpretation
\item
  CTMC transition diagrams and modeling examples
\item
  Steady-state analysis and long-run behavior
\item
  Solving for steady-state probabilities
\item
  Applications to customer flow and occupancy in small systems
\end{itemize}

\subsection{Section 3: Queueing Theory
Essentials}\label{section-3-queueing-theory-essentials}

\begin{itemize}
\tightlist
\item
  What is a queueing system? Components and notation (A/B/S/d/e)
\item
  The Poisson process and M/M/1 queue
\item
  Birth-death process interpretation
\item
  Generator matrices for queues
\item
  Traffic intensity, stability conditions
\item
  Key performance metrics
\item
  Little's Law and system design
\end{itemize}

\subsection{Section 4: Vaccine Clinic Case
Study}\label{section-4-vaccine-clinic-case-study}

\begin{itemize}
\tightlist
\item
  CTMC model for an observation room
\item
  Scenario-based analysis and service optimization
\item
  R-based lab activity: matrix methods, simulation, and performance
  metrics
\item
  Translating modeling results to logistics planning
\end{itemize}




\end{document}
